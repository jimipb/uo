\chapter{Einleitung}

Im Laufe der Entwicklung moderner Halbleitertechnik werden immer neue Techniken
und Materialien nötig, um die Leistungsfähigkeit und Effizienz noch weiter zu
steigern. Seit einiger Zeit stellt das beliebte Halbleitermaterial Silizium selbst
eine Hürde dar. Auf der Suche nach geeigneten Alternativen ist es nötig die 
Prozesse im Festkörper auf ultrakurzen Zeitskalen zu verstehen. Hierzu mussten
neue Methoden entwickelt werden, um eine zeitliche Auflösung bis in den 
Femtosekundenbereich überhaupt möglich zu machen. Dieser Versuch soll eine 
kleine Einführung in die Thematik bieten.
