\chapter{Fazit}

Der Versuch ist eine gute Einführung in die Thematik der ultra schnellen Optik. Es werden interessante Einblicke in die Verwendung von gepulsten Lasern und in neue Experimentiermethoden wie z.B. die Autokorrelation oder das Pump-Probe-Verfahren gegeben. Die Justage der Laserstrahlen bzw. der Proben oder Photodioden gestalten sich oft etwas zeitintensiv und feinfühlig, was jedoch schon aus dem Versuch "Diodengepumpter Festkörperlaser"{} bekannt war, an welchen sich dieser Versuch gut anschließt. Besonders hervorheben möchten wir die Tatsache, dass sich die Suche nach der zweiten Harmonischen im ersten Versuchsteil unter Verwendung der Laserschutzbrillen durchaus schwierig gestaltet, da diese auch die zweite Harmonische des Laser herausfiltern und somit unsichtbar machen. 
	
